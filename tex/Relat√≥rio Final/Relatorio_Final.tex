\documentclass[
	12pt,				% tamanho da fonte
	openright,			% capítulos começam em pág ímpar (insere página vazia caso preciso)
	twoside,			% para impressão em recto e verso. Oposto a oneside
	a4paper,			% tamanho do papel. 
	english,			% idioma adicional para hifenização
	french,				% idioma adicional para hifenização
	spanish,			% idioma adicional para hifenização
	brazil				% o último idioma é o principal do documento
	]{abntex2}


\usepackage{lmodern}			% Usa a fonte Latin Modern			
\usepackage[T1]{fontenc}		% Selecao de codigos de fonte.
\usepackage[utf8]{inputenc}		% Codificacao do documento (conversão automática dos acentos)
\usepackage{lastpage}			% Usado pela Ficha catalográfica
\usepackage{indentfirst}		% Indenta o primeiro parágrafo de cada seção.
\usepackage{color}				% Controle das cores
\usepackage{graphicx}			% Inclusão de gráficos
\usepackage{microtype} 			% para melhorias de justificação
\usepackage{amsmath}
\usepackage{amsfonts}
% ---
\usepackage{lipsum}				% para geração de dummy text
% ---

% ---
% Pacotes de citações
% ---
\usepackage[brazilian,hyperpageref]{backref}	 % Paginas com as citações na bibl
\usepackage[alf]{abntex2cite}	% Citações padrão ABNT

% --- 
% CONFIGURAÇÕES DE PACOTES
% --- 

% ---
% Configurações do pacote backref
% Usado sem a opção hyperpageref de backref
\renewcommand{\backrefpagesname}{Citado na(s) página(s):~}
% Texto padrão antes do número das páginas
\renewcommand{\backref}{}
% Define os textos da citação
\renewcommand*{\backrefalt}[4]{
	\ifcase #1 %
		Nenhuma citação no texto.%
	\or
		Citado na página #2.%
	\else
		Citado #1 vezes nas páginas #2.%
	\fi}%
% ---

% ---
% Informações de dados para CAPA e FOLHA DE ROSTO
% ---
\titulo{Adaptação do algoritmo \textit{Particle Swarm Optimization} (PSO) para identificação de \textit{clusters} espaciais}
\autor{Augusto Cesar Ribeiro Nunes}
\local{Brasília, Brasil}
\data{Junho de 2017}
\orientador{André Luiz Fernandes Cançado}
\instituicao{%
  Universidade de Brasília
  \par
  Departamento de Estatística
  \par
  Graduação}
\tipotrabalho{Trabalho de Conclusão de Curso (Graduação)}
% O preambulo deve conter o tipo do trabalho, o objetivo, 
% o nome da instituição e a área de concentração 
\preambulo{Trabalho de Conclusão de Curso apresentado para obtenção do título de Bacharel em Estatística.}
% ---


% ---
% Configurações de aparência do PDF final

% alterando o aspecto da cor azul
\definecolor{blue}{RGB}{41,5,195}

% informações do PDF
\makeatletter
\hypersetup{
     	%pagebackref=true,
		pdftitle={\@title}, 
		pdfauthor={\@author},
    	pdfsubject={\imprimirpreambulo},
	    pdfcreator={LaTeX with abnTeX2},
		pdfkeywords={abnt}{latex}{abntex}{abntex2}{trabalho acadêmico}, 
		colorlinks=true,       		% false: boxed links; true: colored links
    	linkcolor=blue,          	% color of internal links
    	citecolor=blue,        		% color of links to bibliography
    	filecolor=magenta,      		% color of file links
		urlcolor=blue,
		bookmarksdepth=4
}
\makeatother
% --- 

% --- 
% Espaçamentos entre linhas e parágrafos 
% --- 

% O tamanho do parágrafo é dado por:
\setlength{\parindent}{1.3cm}

% Controle do espaçamento entre um parágrafo e outro:
\setlength{\parskip}{0.2cm}  % tente também \onelineskip

% ---
% compila o indice
% ---
\makeindex
% ---

% ----
% Início do documento
% ----
\begin{document}

% Seleciona o idioma do documento (conforme pacotes do babel)
%\selectlanguage{english}
\selectlanguage{brazil}

% Retira espaço extra obsoleto entre as frases.
\frenchspacing 

% ----------------------------------------------------------
% ELEMENTOS PRÉ-TEXTUAIS
% ----------------------------------------------------------
% \pretextual

% ---
% Capa
% ---
\imprimircapa
% ---

% ---
% Folha de rosto
% (o * indica que haverá a ficha bibliográfica)
% ---
\imprimirfolhaderosto*
% ---

% ---
% Inserir a ficha bibliografica
% ---

% Isto é um exemplo de Ficha Catalográfica, ou ``Dados internacionais de
% catalogação-na-publicação''. Você pode utilizar este modelo como referência. 
% Porém, provavelmente a biblioteca da sua universidade lhe fornecerá um PDF
% com a ficha catalográfica definitiva após a defesa do trabalho. Quando estiver
% com o documento, salve-o como PDF no diretório do seu projeto e substitua todo
% o conteúdo de implementação deste arquivo pelo comando abaixo:
%
% \begin{fichacatalografica}
%     \includepdf{fig_ficha_catalografica.pdf}
% \end{fichacatalografica}

\begin{fichacatalografica}
	\sffamily
	\vspace*{\fill}					% Posição vertical
	\begin{center}					% Minipage Centralizado
	\fbox{\begin{minipage}[c][8cm]{13.5cm}		% Largura
	\small
	\imprimirautor
	%Sobrenome, Nome do autor
	
	\hspace{0.5cm} \imprimirtitulo  / \imprimirautor. --
	\imprimirlocal, \imprimirdata-
	
	\hspace{0.5cm} \pageref{LastPage} p. : il. (algumas color.) ; 30 cm.\\
	
	\hspace{0.5cm} \imprimirorientadorRotulo~\imprimirorientador\\
	
	\hspace{0.5cm}
	\parbox[t]{\textwidth}{\imprimirtipotrabalho~--~\imprimirinstituicao,
	\imprimirdata.}\\
	
	\hspace{0.5cm}
		1. Estatística Espacial.
		2. Detecção de Conglomerados.
		2. Particle Swarm Optimization.
		I. André Luiz Fernandes Cançado.
		II. Universidade de Brasília.
		III. Departamento de Estatística.
		IV. Adaptação do algoritmo \textit{Particle Swarm Optimization} (PSO) para identificação de \textit{clusters} espaciais		
	\end{minipage}}
	\end{center}
\end{fichacatalografica}
% ---

% ---
% Inserir errata
% ---
%\begin{errata}
%Elemento opcional da \citeonline[4.2.1.2]{NBR14724:2011}. Exemplo:
%
%\vspace{\onelineskip}
%
%FERRIGNO, C. R. A. \textbf{Tratamento de neoplasias ósseas apendiculares com
%reimplantação de enxerto ósseo autólogo autoclavado associado ao plasma
%rico em plaquetas}: estudo crítico na cirurgia de preservação de membro em
%cães. 2011. 128 f. Tese (Livre-Docência) - Faculdade de Medicina Veterinária e
%Zootecnia, Universidade de São Paulo, São Paulo, 2011.
%
%\begin{table}[htb]
%\center
%\footnotesize
%\begin{tabular}{|p{1.4cm}|p{1cm}|p{3cm}|p{3cm}|}
%  \hline
%   \textbf{Folha} & \textbf{Linha}  & \textbf{Onde se lê}  & \textbf{Leia-se}  \\
%    \hline
%    1 & 10 & auto-conclavo & autoconclavo\\
%   \hline
%\end{tabular}
%\end{table}
%
%\end{errata}
% ---

% ---
% Inserir folha de aprovação
% ---

% Isto é um exemplo de Folha de aprovação, elemento obrigatório da NBR
% 14724/2011 (seção 4.2.1.3). Você pode utilizar este modelo até a aprovação
% do trabalho. Após isso, substitua todo o conteúdo deste arquivo por uma
% imagem da página assinada pela banca com o comando abaixo:
%
% \includepdf{folhadeaprovacao_final.pdf}
%
%\begin{folhadeaprovacao}
%
%  \begin{center}
%    {\ABNTEXchapterfont\large\imprimirautor}
%
%    \vspace*{\fill}\vspace*{\fill}
%    \begin{center}
%      \ABNTEXchapterfont\bfseries\Large\imprimirtitulo
%    \end{center}
%    \vspace*{\fill}
%    
%    \hspace{.45\textwidth}
%    \begin{minipage}{.5\textwidth}
%        \imprimirpreambulo
%    \end{minipage}%
%    \vspace*{\fill}
%   \end{center}
%        
%   Trabalho aprovado. \imprimirlocal, 24 de novembro de 2012:
%
%   \assinatura{\textbf{\imprimirorientador} \\ Orientador} 
%   \assinatura{\textbf{Professor} \\ Convidado 1}
%   \assinatura{\textbf{Professor} \\ Convidado 2}
%   %\assinatura{\textbf{Professor} \\ Convidado 3}
%   %\assinatura{\textbf{Professor} \\ Convidado 4}
%      
%   \begin{center}
%    \vspace*{0.5cm}
%    {\large\imprimirlocal}
%    \par
%    {\large\imprimirdata}
%    \vspace*{1cm}
%  \end{center}
%  
%\end{folhadeaprovacao}
% ---

% ---
% Dedicatória
% ---
%\begin{dedicatoria}
%   \vspace*{\fill}
%   \centering
%   \noindent
%   \textit{ Este trabalho é dedicado às crianças adultas que,\\
%   quando pequenas, sonharam em se tornar cientistas.} \vspace*{\fill}
%\end{dedicatoria}
% ---

% ---
% Agradecimentos
% ---
%\begin{agradecimentos}
%Os agradecimentos principais são direcionados à Gerald Weber, Miguel Frasson,
%Leslie H. Watter, Bruno Parente Lima, Flávio de Vasconcellos Corrêa, Otavio Real
%Salvador, Renato Machnievscz\footnote{Os nomes dos integrantes do primeiro
%projeto abn\TeX\ foram extraídos de
%\url{http://codigolivre.org.br/projects/abntex/}} e todos aqueles que
%contribuíram para que a produção de trabalhos acadêmicos conforme
%as normas ABNT com \LaTeX\ fosse possível.
%
%Agradecimentos especiais são direcionados ao Centro de Pesquisa em Arquitetura
%da Informação\footnote{\url{http://www.cpai.unb.br/}} da Universidade de
%Brasília (CPAI), ao grupo de usuários
%\emph{latex-br}\footnote{\url{http://groups.google.com/group/latex-br}} e aos
%novos voluntários do grupo
%\emph{\abnTeX}\footnote{\url{http://groups.google.com/group/abntex2} e
%\url{http://www.abntex.net.br/}}~que contribuíram e que ainda
%contribuirão para a evolução do \abnTeX.
%
%\end{agradecimentos}
% ---

% ---
% Epígrafe
% ---
%\begin{epigrafe}
%    \vspace*{\fill}
%	\begin{flushright}
%		\textit{``Não vos amoldeis às estruturas deste mundo, \\
%		mas transformai-vos pela renovação da mente, \\
%		a fim de distinguir qual é a vontade de Deus: \\
%		o que é bom, o que Lhe é agradável, o que é perfeito.\\
%		(Bíblia Sagrada, Romanos 12, 2)}
%	\end{flushright}
%\end{epigrafe}
% ---

% ---
% RESUMOS
% ---

% resumo em português
\setlength{\absparsep}{18pt} % ajusta o espaçamento dos parágrafos do resumo
\begin{resumo}
Este Trabalho de Conclusão de Curso contrapõe o \textit{Scan} Circular de Kulldorff e uma implementação do BPSO para a Análise e Identificação de \textit{Clusters} em um mapa de 203 hexágonos regulares e avalia ambos quanto ao seu Poder, Sensibilidade e Valor Preditivo Positivo.

  \textbf{Palavras-chave}: Estatística Espacial. \textit{Particle Swarm Optimization}. \textit{Binary Particle Swarm Optimization}. 
%  Segundo a \citeonline[3.1-3.2]{NBR6028:2003}, o resumo deve ressaltar o
%  objetivo, o método, os resultados e as conclusões do documento. A ordem e a extensão
%  destes itens dependem do tipo de resumo (informativo ou indicativo) e do
%  tratamento que cada item recebe no documento original. O resumo deve ser
%  precedido da referência do documento, com exceção do resumo inserido no
%  próprio documento. (\ldots) As palavras-chave devem figurar logo abaixo do
%  resumo, antecedidas da expressão Palavras-chave:, separadas entre si por
%  ponto e finalizadas também por ponto.

%  \textbf{Palavras-chave}: latex. abntex. editoração de texto.
\end{resumo}

% resumo em inglês
\begin{resumo}[Abstract]
 \begin{otherlanguage*}{english}
   This is the english abstract.

   \vspace{\onelineskip}
 
   \noindent 
   \textbf{Keywords}: latex. abntex. text editoration.
 \end{otherlanguage*}
\end{resumo}

% % resumo em francês 
% \begin{resumo}[Résumé]
%  \begin{otherlanguage*}{french}
%     Il s'agit d'un résumé en français.
 
%    \textbf{Mots-clés}: latex. abntex. publication de textes.
%  \end{otherlanguage*}
% \end{resumo}

% % resumo em espanhol
% \begin{resumo}[Resumen]
%  \begin{otherlanguage*}{spanish}
%    Este es el resumen en español.
  
%    \textbf{Palabras clave}: latex. abntex. publicación de textos.
%  \end{otherlanguage*}
% \end{resumo}
% ---

% ---
% inserir lista de ilustrações
% ---
\pdfbookmark[0]{\listfigurename}{lof}
\listoffigures*
\cleardoublepage
% ---

% ---
% inserir lista de tabelas
% ---
\pdfbookmark[0]{\listtablename}{lot}
\listoftables*
\cleardoublepage
% ---

% ---
% inserir lista de abreviaturas e siglas
% ---
\begin{siglas}
  \item[ABNT] Associação Brasileira de Normas Técnicas
  \item[abnTeX] ABsurdas Normas para TeX
\end{siglas}
% ---

% ---
% inserir lista de símbolos
% ---
\begin{simbolos}
  \item[$ \Gamma $] Letra grega Gama
  \item[$ \Lambda $] Lambda
  \item[$ \zeta $] Letra grega minúscula zeta
  \item[$ \in $] Pertence
\end{simbolos}
% ---

% ---
% inserir o sumario
% ---
\pdfbookmark[0]{\contentsname}{toc}
\tableofcontents*
\cleardoublepage
% ---



% ----------------------------------------------------------
% ELEMENTOS TEXTUAIS
% ----------------------------------------------------------
\textual

% ----------------------------------------------------------
% Introdução (exemplo de capítulo sem numeração, mas presente no Sumário)
% ----------------------------------------------------------
\chapter*[Introdução]{Introdução}
\addcontentsline{toc}{chapter}{Introdução}
% ----------------------------------------------------------


Se conceituarmos, ingenuamente, a Estatística como um compêndio de metodologias que tem como objetivo a descrição da variabilidade presente em processos, podemos entender a Estatística Espacial como tão somente a aplicação de metodologias adequadas a observações distribuídas de maneira espacial. 

Seu desenvolvimento como área da Estatística Moderna acompanhou intimamente e de maneira concomitante estudos em diversas áreas do conhecimento e, como em inúmeras outras aplicações, foi fortemente impulsionado pelo aumento significativo tanto da disponibilidade quanto do poder computacional nas últimas décadas. 

Há ainda a importante distinção entre problemas meramente espaciais e aqueles onde o tempo é uma dimensão não-desprezível. Processos espaço-temporais são, via de regra, tratados de maneira diferente em relação àqueles sem a presença da componente temporal. Este trabalho lida com um problema não-estocástico, ainda que o mesmo possa ser adaptado de forma a acomodar adequadamente a dimensão temporal.

Como tema em Estatística Espacial, o problema de análise de \textit{clusters} consiste em, supondo um certo processo aleatório que ocorre em um espaço delimitado, avaliar se o mesmo ocorre de maneira meramente aleatória ou se há a presença de um padrão definidor da presença de um conglomerado, ou seja, um \textit{cluster}. A etapa de identificação consiste em descrever, sob o sistema de referência, qual é o conglomerado identificado. As heurísticas dedicam-se primariamente à etapa de análise do conglomerado, à investigação de sua presença ou ausência, sendo a etapa de identificação executada em um segundo momento e decorrendo da primeira. Uma definição formal e útil é apresentada em \ref{sec:geral}, o caso particular deste Trabalho de Conclusão de Curso é dada em \ref{sec:dmg}.

Uma heurística tradicional para a detecção de \textit{clusters} é o Algoritmo do \textit{Scan} Circular de Kulldorff, proposto em \cite{kulldorff1997spatial}, descrito em maiores detalhes sob sua forma não-estocástica em \ref{sec:ksc}. Este algoritmo simples e versátil, e cujo uso perpassa diversas áreas da ciência aplicada \cite{kulldorff1999spatial}, sofre de uma limitação potencialmente grave sugerida pelo seu próprio nome: o formato circular de sua janela de análise pode diminuir seu poder na detecção de \textit{clusters} não-regulares. O \textit{Scan} Circular de Kulldorff é a primeira metodologia de identificação de \textit{clusters} implementada neste trabalho.

Deixando temporariamente a Estatística de lado e voltando os olhos para a Ciência da Computação, na década de 1960 quatro metodologias diferentes foram desenvolvidas de maneira mais ou menos independente e compõem o paradigma que hoje é conhecido como Computação Evolucionária. Como o nome sugere, as heurísticas desse tipo inspiram-se em maior ou menor grau nos mecanismos evolutivos da natureza: conceitos como "população", "seleção", "competição", "mutação" etc. são encontrados em abundância na literatura da área, como descrito em \cite{fogel1994introduction} e \cite{back1997evolutionary}. Em termos gerais, a motivação por trás de tais desenvolvimentos foi a demanda crescente por sistemas com um certo nível de autonomia, ou que disponham de um certo nível de \textit{inteligência}, ainda que rudimentar (dada a escassa capacidade computacional da época) e que algumas abordagens façam um contraponto entre "evolução" e "inteligência". 

Uma das aplicações mais promissoras dos algoritmos desenvolvidos sob a perspectiva da Computação Evolucionária é a de problemas de otimização. Destas, uma heurística muito geral e poderosa envolve o que os autores chamaram de "Enxame de Partículas" (\textit{Particle Swarm}), que daria origem ao Algoritmo \textit{Particle Swarm Optimization}, ou PSO \cite{eberhart1995new}. O PSO também empresta termos que até então eram utilizados em outros contextos, e sua motivação admitidamente advém do vôo de pássaros e peixes em cardume. Decididamente, em sua aplicação o algoritmo provou-se uma heurística versátil e poderosa em uma grande gama de problemas, como mostra \cite{shi2001particle}. O PSO é descrito com maiores detalhes em \ref{chap:PSO}.

Entretanto, uma limitação da proposição original do PSO era a de que aplicava-se apenas a contextos de otimização real, ou contínua. Em \cite{kennedy1997discrete} os próprios autores da heurística original propuseram adaptações que possibilitaram que o mesmo pudesse ser utilizado em contextos de otimização discreta, a este algoritmo adaptaram deram o nome de \textit{Binary Particle Swarm Optimization}, ou BPSO. O BPSO é a segunda metodologia de detecção de \textit{clusters} aqui implementada: espera-se que ele seja mais versátil na detecção de \textit{clusters} com formatos \textit{irregulares} quando comparado ao \textit{Scan} Circular de Kulldorff, e que não seja inaceitavelmente pior que ele quando da detecção de \textit{clusters} circulares.

O mapa composto por 203 hexágonos utilizado aqui é descrito em \ref{sec:mapa}. A distribuição dos 406 casos é feita de 4 maneiras diferentes, cada uma representa um formato de \textit{cluster} conhecido e que desejamos que os algoritmos sejam capazes de detectá-los. Estes 4 cenários são descritos em \ref{sec:casos}

Os procedimentos de simulação utilizados envolvem Simulação de Monte Carlo para o \textit{Scan} Circular de Kulldorff e o BPSO, descritos em \ref{sec:simul}. 

% ----------------------------------------------------------
% PARTE
% ----------------------------------------------------------
\part{A análise e identificação de \textit{clusters}}
\label{parte:1}

\chapter{Descrição do problema de análise de \textit{clusters}}
\label{chap:prob} 




\section{De maneira geral}
\label{sec:geral}

\cite{cressie2015statistics} conceitua três tipos de dados mais comumente encontrados nos problemas de Estatística Espacial:

\begin{itemize}
	\item Dados pontuais: Coordenadas $(X_i, Y_i)$ de ocorrência ou presença de cada i-ésimo evento; 
    \item Dados de área: Não dispomos da referência exata do evento mas sim de dados de contagem do mesmo em uma certa região do Espaço Geográfico;
    \item Dados de superfície: A tripla $(X_i, Y_i, Z_i)$ indica, além das coordenadas de ocorrência do i-ésimo evento, uma medição $Z_i$ do mesmo.
\end{itemize}

Claramente, estas três estruturas de dados apresentam uma certa equivalência e eventualmente - de acordo com a natureza do problema - pode ser mais indicado utilizar uma ou outra. 

Resgatando a \textit{definição} informal dada na Introdução do trabalho: a análise de \textit{clusters} dedica-se à avaliação de um certo fenômeno que ocorre no espaço quanto à homogeneidade do mesmo. Supondo que dispomos de dados pontuais de ocorrência, uma das abordagens possíveis para este problema é apresentada em \cite{diggle2013statistical}, sob o que ele chama de \textit{Complete Spatial Randomness Hypothesis}, ou Hipótese de Aleatoriedade Espacial Completa, que prescreve duas suposições:

\begin{enumerate}
\item O número de eventos em um região planar A de área $|A|$ segue uma distribuição de Poisson com média $\lambda |A|$;
\item Dados $n$ eventos $x_i$ na região A, os $x_i$ são uma amostra aleatória da distribuição uniforme em A.
\end{enumerate}

A abordagem consiste então em testar essa hipótese, em particular suas suposições. O foco principal neste trabalho está em testar a primeira hipótese, de que a \textit{intensidade} $\lambda$ de ocorrência é constante. A hipótese CSR é apresentada como uma abordagem inicial a ser realizada ainda na fase exploratória do estudo.
% Ao lidar com dados pontuais ou de área, uma pergunta pode surgir: os eventos em questão ocorrem de maneira \textit{aleatória} ou há um padrão detectável? Uma definição formal e breve, presente em \cite{kulldorff1997spatial}: sob o Espaço Geográfico G, supomos um processo pontual espacial aleatório N, onde N(A) é o número aleatório de pontos no conjunto $A \in G$. \cite{diggle2013statistical} define processo pontual espacial aleatório como um mecanismo estocástico que gera um conjunto contável de eventos no plano.

\section{Neste trabalho}
\label{sec:dmg}



\part{O Algoritmo \textit{Particle Swarm Optimization} (PSO) e sua variante Binária}
\chapter{Conceituação da Versão Tradicional do Algoritmo}
% ---

% ---
\section{Origem}

O algoritmo PSO é uma técnica de otimização estocástica baseada em populações originalmente proposta em \cite{eberhart1995new}. Sua tradução como \textit{Otimização usando Enxame de Partículas} sugere sua motivação metafórica no comportamento observado em um modelo social, particularmente aves em revoada à procuar de comida.  

\section{Contexto de Otimização}
Supondo que o problema alvo tenha $n$ dimensões e uma população de partículas, que representam soluções para o problema e movem-se em um espaço de busca a procura de soluções melhores. Cada partícula tem um vetor de posição $\boldmath{X}_i$ e um vetor de velocidade $\boldmath{V}_i$, que são representados no espaço n-dimensional como $\boldmath{X}_i = (x_{i1}, x_{i2}, \dots, x_{in})$ e  $\boldmath{V}_i = (v_{i1}, v_{i2}, \dots, v_{in})$, respectivamente, e também uma memória da melhor posição que ocupou no espaço de busca até o presente momento $(\text{Pbest}_i)$ bem como a melhor localização encontrada até então por todas as partículas no \textit{enxame} $(\text{Gbest})$, que são representadas no espaço n-dimensional como $\text{Pbest}_i = (x_{i1}^{Pbest}, x_{i2}^{Pbest}, \dots, x_{in}^{Pbest})$ e $\text{Gbest} = (x_1^{Gbest}, x_2^{Gbest}, \dots, x_n^{Gbest})$, respectivamente. 

A cada iteração, a velocidade da i-ésima partícula é atualizada de acordo com a seguinte equação no algoritmo PSO:

\begin{equation}
\boldmath{V}_i^{k+1} = \omega\boldmath{V}_i^k + c_1r_1 \times (\text{Pbest}_i^k - \boldmath{X}_i^k) + c_2r_2\times(\text{Gbest}^k - \boldmath{X}_i^k)
\label{eq:PSO1}
\end{equation}

onde,

\begin{itemize}
\item $\boldmath{V}_i^{k+1}$ é a velocidade da partícula i na iteração $k+1$,
\item $\boldmath{V}_i^{k}$ é a velocidade da partícula i na iteração $k$,
\item $\omega$ é o parâmetro de peso inercial,
\item $c_1$ e $c_2$ são coeficientes de aceleração,
\item $r_1$ e $r_2$ são números aleatórios entre 0 e 1,
\item $\boldmath{X}_i^k$ é a posição da partícula i na iteração $k$,
\item $\text{Pbest}_i^k$ é a melhor posição da partícula i até a iteração $k$,
\item $\text{Gbest}^k$ é a melhor posição global das partículas até a iteracão $k$.
\end{itemize}

Em \cite{kennedy1997particle}, o autor apresenta uma interpretação do algoritmo, mais especificamente da equação \ref{eq:PSO1}, sob a perspectiva da Psicologia Social separando-a em dois termos:

\begin{itemize}
\item O primeiro termo, chamado de "Termo Cognitivo", baseia-se na noção de que comportamentos aleatórios que são seguidos por reforços positivos tornam-se mais prováveis no futuro;
\item O segundo termo, chamado de "Termo Social", fundamenta-se no conceito de que a observação de um comportamento cuja cognição é correta, válida ou consistente deve resultar na imitação deste comportamento pelo observador.
\end{itemize}

Neste processo de atualização de velocidades, os coeficientes de aceleração estocástica $c_1$ e $c_2$ e o peso inercial $\omega$ são pré-definidos e são responsáveis por acelerar a partícula na direção de $\text{Pbest}_i^k$ e $\text{Gbest}^k$ respectivamente: valores pequenos para estes coeficientes fazem com quem as partículas percorram trajetórias distantes dos pontos ótimos - locais ou global - e valores grandes resultam em movimentos abruptos na direção dos pontos ótimos. Uma escolha adequada do peso inercial $\omega$ provê um equilíbrio entre explorações locais e globais das partículas, reduzindo assim o número necessário de iterações até que um ponto de ótimo seja encontrado. Uma das escolhas, proposta em \cite{kennedy2001swarm} e chamada de \textit{Inertia Weight Approach} (IWA), consiste em defini-lo da seguinte maneira:

\begin{equation}
\omega = \omega_{\text{max}} - \frac{\omega_{\text{max}} - \omega_{\text{min}}}{\text{Iter}_{\text{max}}} \times \text{Iter}
\label{eq:pesoinercial}
\end{equation}

onde

\begin{itemize}
\item $\omega_{\text{max}}$ e $\omega_{\text{min}}$ são os pesos inerciais final e inicial, respectivamente, 
\item $\text{Iter}_{\text{max}}$ é o número máximo de iterações,
\item $\text{Iter}$ é o número atual da iteração.
\end{itemize}


Cada indivíduo movimenta-se da posição atual para a seguinte com a velocidade modificada \ref{eq:PSO1} usando a equação

\begin{equation}
\boldmath{X}_i^{k+1} = \boldmath{X}_i^{k} + \boldmath{V}_i^{k+1}
\label{eq:x_at}
\end{equation}




\section{Abordagem utilizando um Fator de Constrição K}

Após a apresentação inicial do algoritmo uma série de abordagens derivadas foram propostas, no entanto sempre sob a perspectiva empírica da aplicação da heurística em certos contextos \cite{van2001analysis}. Um dos estudos que se debruçam sobre a fundamentação matemática do algoritmo é \cite{clerc2002particle}. A abordagem deste trabalho trata as equações \ref{eq:PSO1}, \ref{eq:pesoinercial} e \ref{eq:x_at} como equações de diferenças, e portanto reduzindo o procedimento de busca do algoritmo a um problema de análise de autovalores de forma a garantir sua convergência em termos globais e evitar a convergência "prematura" para ótimos locais. Uma das conclusões dos autores foi a de que o PSO sem o Termo Social apresenta uma performance inaceitável, ou seja, o aspecto colaborativo do algoritmo é fundamental para sua eficiência; a segunda e talvez mais importante é a de que para garantir as características de convergência citadas - a "explosão" do algoritmo era até então controlada utilizando uma restrição $V_{max}$ - é necessária a introdução do que chamaram de "Fator de Constrição K", que modifica a equação \ref{eq:PSO1} da seguinte forma:

\begin{equation}
\boldmath{V}_i^{k+1} = K\left[\omega\boldmath{V}_i^k + c_1r_1 \times (\text{Pbest}_i^k - \boldmath{X}_i^k) + c_2r_2\times(\text{Gbest}^k - \boldmath{X}_i^k)\right]
\label{eq:PSO1FC}
\end{equation}

com

\begin{equation*}
 K = \frac{2}{|2 - \varphi - \sqrt{\varphi^2 - 4\varphi}|}\text{,	}  \varphi = c_1 + c_2, \varphi > 4
\end{equation*}

Os autores mostraram que \ref{eq:PSO1FC} resulta em garantia de convergência, ao contrário da abordagem que utilizava a restrição $V_{max}$. 

\chapter{O PSO binário}

Em \cite{kennedy1997discrete} os autores, motivados pela presença de inúmeros contextos onde a otimização é feita em espaços discretos, exigem algumas modificações, notadamente:

\begin{enumerate}
	\item Começando pelo espaço de busca, que deixa de ser n-dimensional em $\mathbb{R}^n$ e passa a ser um hipercubo, onde a posição da partícula é representada por uma \textit{string} (ou um vetor) de \textit{bits}, atualizada a cada iteração.
	\item A velocidade da partícula deixa de ser descrita por \ref{eq:PSO1} ou \ref{eq:PSO1FC} e similares, e passa a ser descrita pela Distância de Hamming, ou o número de \textit{bits} modificados na \textit{string} de \textit{bits} da partícula a cada iteração: uma partícula com \textit{velocidade} nula neste contexto é aquela cuja \textit{string} de posição não teve nenhum \textit{bit} modificado entre as iterações $t$ e $t + 1$, da mesma forma que uma partícula com velocidade máxima é aquela onde todos os \textit{bits} de seu vetor de posições foram modificados
\end{enumerate}  




\begin{table}[htb]
\IBGEtab{%
  \caption{Ilustração de diferenças entre atributos do PSO original e PSO binário.}%
  \label{tabela-ibge}
}{%
  \begin{tabular}{ccc}
  \toprule
   Atributo & PSO Original & PSO binário \\
  \midrule \midrule
   Espaço de Busca A & $A \subseteq \mathbb{R}^D$ & $A \subset \{0,1\}^D$ \\
  \midrule
  Posição & Vetor real D-dimensional definido em \ref{eq:x_at} & \textit{String} binária D-dimensional \\
  \midrule
   Velocidade da partícula & Vetor a partir de \ref{eq:PSO1}, \ref{eq:PSO1FC} & Distância de Hamming e similares\\
  \midrule 
   Velocidade de \\ cada componente & Componentes do vetor \ref{eq:PSO1}, \ref{eq:PSO1FC} e similares & \\ 
  \bottomrule
\end{tabular}%
}{%
  \fonte{Produzido pelos autores.}%
  \nota{Esta é uma nota, que diz que os dados são baseados na
  regressão linear.}%
  \nota[Anotações]{Uma anotação adicional, que pode ser seguida de várias
  outras.}%
  }
\end{table}
% ----------------------------------------------------------
% PARTE
% ----------------------------------------------------------
\part{Referenciais teóricos}
% ----------------------------------------------------------

% ---
% Capitulo de revisão de literatura
% ---

% ---

\lipsum[1]

\lipsum[2-3]

% ----------------------------------------------------------
% PARTE
% ----------------------------------------------------------
\part{Resultados}
% ----------------------------------------------------------

% ---
% primeiro capitulo de Resultados
% ---
\chapter{Lectus lobortis condimentum}
% ---

% ---
\section{Vestibulum ante ipsum primis in faucibus orci luctus et ultrices
posuere cubilia Curae}
% ---

\lipsum[21-22]

% ---
% segundo capitulo de Resultados
% ---
\chapter{Nam sed tellus sit amet lectus urna ullamcorper tristique interdum
elementum}
% ---

% ---
\section{Pellentesque sit amet pede ac sem eleifend consectetuer}
% ---

\lipsum[24]

% ----------------------------------------------------------
% Finaliza a parte no bookmark do PDF
% para que se inicie o bookmark na raiz
% e adiciona espaço de parte no Sumário
% ----------------------------------------------------------
\phantompart

% ---
% Conclusão
% ---
\chapter{Conclusão}
% ---

\lipsum[31-33]

% ----------------------------------------------------------
% ELEMENTOS PÓS-TEXTUAIS
% ----------------------------------------------------------
\postextual
% ----------------------------------------------------------

% ----------------------------------------------------------
% Referências bibliográficas
% ----------------------------------------------------------
\bibliography{ref-Relatorio_Final}

% ----------------------------------------------------------
% Glossário
% ----------------------------------------------------------
%
% Consulte o manual da classe abntex2 para orientações sobre o glossário.
%
%\glossary

% ----------------------------------------------------------
% Apêndices
% ----------------------------------------------------------

% ---
% Inicia os apêndices
% ---
\begin{apendicesenv}

% Imprime uma página indicando o início dos apêndices
\partapendices

% ----------------------------------------------------------
\chapter{Quisque libero justo}
% ----------------------------------------------------------

\lipsum[50]

% ----------------------------------------------------------
\chapter{Nullam elementum urna vel imperdiet sodales elit ipsum pharetra ligula
ac pretium ante justo a nulla curabitur tristique arcu eu metus}
% ----------------------------------------------------------
\lipsum[55-57]

\end{apendicesenv}
% ---


% ----------------------------------------------------------
% Anexos
% ----------------------------------------------------------

% ---
% Inicia os anexos
% ---
\begin{anexosenv}

% Imprime uma página indicando o início dos anexos
\partanexos

% ---
\chapter{Morbi ultrices rutrum lorem.}
% ---
\lipsum[30]

% ---
\chapter{Cras non urna sed feugiat cum sociis natoque penatibus et magnis dis
parturient montes nascetur ridiculus mus}
% ---

\lipsum[31]

% ---
\chapter{Fusce facilisis lacinia dui}
% ---

\lipsum[32]

\end{anexosenv}

%---------------------------------------------------------------------
% INDICE REMISSIVO
%---------------------------------------------------------------------
\phantompart
\printindex
%---------------------------------------------------------------------

\end{document}
