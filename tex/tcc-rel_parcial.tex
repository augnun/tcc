%% abtex2-modelo-projeto-pesquisa.tex, v-1.9.6 laurocesar
%% Copyright 2012-2016 by abnTeX2 group at http://www.abntex.net.br/ 
%%
%% This work may be distributed and/or modified under the
%% conditions of the LaTeX Project Public License, either version 1.3
%% of this license or (at your option) any later version.
%% The latest version of this license is in
%%   http://www.latex-project.org/lppl.txt
%% and version 1.3 or later is part of all distributions of LaTeX
%% version 2005/12/01 or later.
%%
%% This work has the LPPL maintenance status `maintained'.
%% 
%% The Current Maintainer of this work is the abnTeX2 team, led
%% by Lauro César Araujo. Further information are available on 
%% http://www.abntex.net.br/
%%
%% This work consists of the files abntex2-modelo-projeto-pesquisa.tex
%% and abntex2-modelo-references.bib
%%

% ------------------------------------------------------------------------
% ------------------------------------------------------------------------
% abnTeX2: Modelo de Projeto de pesquisa em conformidade com 
% ABNT NBR 15287:2011 Informação e documentação - Projeto de pesquisa -
% Apresentação 
% ------------------------------------------------------------------------ 
% ------------------------------------------------------------------------

\documentclass[
	% -- opções da classe memoir --
	12pt,				% tamanho da fonte
	openright,			% capítulos começam em pág ímpar (insere página vazia caso preciso)
	twoside,			% para impressão em recto e verso. Oposto a oneside
	a4paper,			% tamanho do papel. 
	% -- opções da classe abntex2 --
	%chapter=TITLE,		% títulos de capítulos convertidos em letras maiúsculas
	%section=TITLE,		% títulos de seções convertidos em letras maiúsculas
	%subsection=TITLE,	% títulos de subseções convertidos em letras maiúsculas
	%subsubsection=TITLE,% títulos de subsubseções convertidos em letras maiúsculas
	% -- opções do pacote babel --
	english,			% idioma adicional para hifenização
	french,				% idioma adicional para hifenização
	spanish,			% idioma adicional para hifenização
	brazil,				% o último idioma é o principal do documento
	]{abntex2}

% ---
% Formatação de código-fonte
% ---
\usepackage{listingsutf8}

% Altera o nome padrão do rótulo usado no comando \autoref{}
\renewcommand{\lstlistingname}{Código}

% Altera o rótulo a ser usando no elemento pré-textual "Lista de código"
\renewcommand{\lstlistlistingname}{Lista de códigos}

% Configura a ``Lista de Códigos'' conforme as regras da ABNT (para abnTeX2)
\begingroup\makeatletter
\let\newcounter\@gobble\let\setcounter\@gobbletwo
  \globaldefs\@ne \let\c@loldepth\@ne
  \newlistof{listings}{lol}{\lstlistlistingname}
  \newlistentry{lstlisting}{lol}{0}
\endgroup

\renewcommand{\cftlstlistingaftersnum}{\hfill--\hfill}

\let\oldlstlistoflistings\lstlistoflistings
\renewcommand{\lstlistoflistings}{%
   \begingroup%
   \let\oldnumberline\numberline%
   \renewcommand{\numberline}{\lstlistingname\space\oldnumberline}%
   \oldlstlistoflistings%
   \endgroup}
   

\lstset{
language = R,
numbers=left,
tabsize=2,
basicstyle=\footnotesize,
literate=
  {á}{{\'a}}1 {é}{{\'e}}1 {í}{{\'i}}1 {ó}{{\'o}}1 {ú}{{\'u}}1
  {Á}{{\'A}}1 {É}{{\'E}}1 {Í}{{\'I}}1 {Ó}{{\'O}}1 {Ú}{{\'U}}1
  {à}{{\`a}}1 {è}{{\`e}}1 {ì}{{\`i}}1 {ò}{{\`o}}1 {ù}{{\`u}}1
  {À}{{\`A}}1 {È}{{\'E}}1 {Ì}{{\`I}}1 {Ò}{{\`O}}1 {Ù}{{\`U}}1
  {ä}{{\"a}}1 {ë}{{\"e}}1 {ï}{{\"i}}1 {ö}{{\"o}}1 {ü}{{\"u}}1
  {Ä}{{\"A}}1 {Ë}{{\"E}}1 {Ï}{{\"I}}1 {Ö}{{\"O}}1 {Ü}{{\"U}}1
  {â}{{\^a}}1 {ê}{{\^e}}1 {î}{{\^i}}1 {ô}{{\^o}}1 {û}{{\^u}}1
  {Â}{{\^A}}1 {Ê}{{\^E}}1 {Î}{{\^I}}1 {Ô}{{\^O}}1 {Û}{{\^U}}1
  {œ}{{\oe}}1 {Œ}{{\OE}}1 {æ}{{\ae}}1 {Æ}{{\AE}}1 {ß}{{\ss}}1
  {ű}{{\H{u}}}1 {Ű}{{\H{U}}}1 {ő}{{\H{o}}}1 {Ő}{{\H{O}}}1
  {ç}{{\c c}}1 {Ç}{{\c C}}1 {ø}{{\o}}1 {å}{{\r a}}1 {Å}{{\r A}}1
  {€}{{\euro}}1 {£}{{\pounds}}1
}

% ---
% PACOTES
% ---

% ---
% Pacotes fundamentais 
% ---
\usepackage{lmodern}			% Usa a fonte Latin Modern
\usepackage[T1]{fontenc}		% Selecao de codigos de fonte.
\usepackage[utf8]{inputenc}		% Codificacao do documento (conversão automática dos acentos)
\usepackage{indentfirst}		% Indenta o primeiro parágrafo de cada seção.
\usepackage{color}				% Controle das cores
\usepackage{graphicx}			% Inclusão de gráficos
\usepackage{microtype} 			% para melhorias de justificação
\usepackage{colortbl}
\usepackage{csvsimple}
\usepackage{longtable}
% ---

% ---
% Pacotes adicionais, usados apenas no âmbito do Modelo Canônico do abnteX2
% ---
\usepackage{lipsum}				% para geração de dummy text
% ---

% ---
% Pacotes de citações
% ---
\usepackage[brazilian,hyperpageref]{backref}	 % Paginas com as citações na bibl
\usepackage[alf]{abntex2cite}	% Citações padrão ABNT

% --- 
% CONFIGURAÇÕES DE PACOTES
% --- 

% ---
% Configurações do pacote backref
% Usado sem a opção hyperpageref de backref
\renewcommand{\backrefpagesname}{Citado na(s) página(s):~}
% Texto padrão antes do número das páginas
\renewcommand{\backref}{}
% Define os textos da citação
\renewcommand*{\backrefalt}[4]{
	\ifcase #1 %
		Nenhuma citação no texto.%
	\or
		Citado na página #2.%
	\else
		Citado #1 vezes nas páginas #2.%
	\fi}%
% ---

% ---
% Informações de dados para CAPA e FOLHA DE ROSTO
% ---
\titulo{Adaptação do algoritmo \textit{Particle Swarm Optimization} (PSO) para identificação de \textit{clusters} espaciais \\ Relatório Parcial}
\autor{Augusto Cesar Ribeiro Nunes}
\local{Brasília, Brasil}
\data{Novembro de 2016}
\instituicao{%
  Universidade de Brasília
  \par
  Departamento de Estatística
  \par
  Graduação}
\tipotrabalho{Trabalho de Conclusão de Curso (Graduação)}
% O preambulo deve conter o tipo do trabalho, o objetivo, 
% o nome da instituição e a área de concentração 
\preambulo{Relatório Parcial do Trabalho de Conclusão de Curso a ser apresentado para obtenção do título de Bacharel em Estatística. \newline \newline Orientador: Prof. Dr. \textbf{André Luiz Fernandes Cançado}}
% ---

% ---
% Configurações de aparência do PDF final

% alterando o aspecto da cor azul
\definecolor{blue}{RGB}{41,5,195}

% informações do PDF
\makeatletter
\hypersetup{
     	%pagebackref=true,
		pdftitle={\@title}, 
		pdfauthor={\@author},
    	pdfsubject={\imprimirpreambulo},
	    pdfcreator={LaTeX with abnTeX2},
		pdfkeywords={tcc}{trabalho de conclusão de curso}{estatística espacial}{estatística computacional}{Universidade de Brasília}{projeto de pesquisa}, 
		colorlinks=true,       		% false: boxed links; true: colored links
    	linkcolor=blue,          	% color of internal links
    	citecolor=blue,        		% color of links to bibliography
    	filecolor=magenta,      		% color of file links
		urlcolor=blue,
		bookmarksdepth=4
}
\makeatother
% --- 

% --- 
% Espaçamentos entre linhas e parágrafos 
% --- 

% O tamanho do parágrafo é dado por:
\setlength{\parindent}{1.3cm}

% Controle do espaçamento entre um parágrafo e outro:
\setlength{\parskip}{0.2cm}  % tente também \onelineskip

% ---
% compila o indice
% ---
\makeindex
% ---

% ----
% Início do documento
% ----
\begin{document}

% Seleciona o idioma do documento (conforme pacotes do babel)
%\selectlanguage{english}
\selectlanguage{brazil}

% Retira espaço extra obsoleto entre as frases.
\frenchspacing 

% ----------------------------------------------------------
% ELEMENTOS PRÉ-TEXTUAIS
% ----------------------------------------------------------
% \pretextual

% ---
% Capa
% ---
\imprimircapa
% ---

% ---
% Folha de rosto
% ---
\imprimirfolhaderosto
% ---

% ---
% NOTA DA ABNT NBR 15287:2011, p. 4:
%  ``Se exigido pela entidade, apresentar os dados curriculares do autor em
%     folha ou página distinta após a folha de rosto.''
% ---

% ---
% inserir lista de ilustrações
% ---
%\pdfbookmark[0]{\listfigurename}{lof}
%\listoffigures*
\cleardoublepage
% ---

% ---
% inserir lista de tabelas
% ---
%\pdfbookmark[0]{\listtablename}{lot}
%\listoftables*
\cleardoublepage
% ---

% ---
% inserir lista de abreviaturas e siglas
% ---
%\begin{siglas}
 % \item[PSO] \textit{Particle Swarm Optimization}
  %\item[abnTeX] ABsurdas Normas para TeX
%\end{siglas}
% ---

% ---
% inserir lista de símbolos
% ---
%\begin{simbolos}
%  \item[$ \Gamma $] Letra grega Gama
%  \item[$ \Lambda $] Lambda
%  \item[$ \zeta $] Letra grega minúscula zeta
%  \item[$ \in $] Pertence
%\end{simbolos}
% ---

% ---
% inserir o sumario
% ---
%\pdfbookmark[0]{\contentsname}{toc}
%\tableofcontents*
\cleardoublepage
% ---


% ----------------------------------------------------------
% ELEMENTOS TEXTUAIS
% ----------------------------------------------------------
\textual

% ----------------------------------------------------------
% Introdução
% ----------------------------------------------------------
\chapter*[Introdução]{Introdução}
\addcontentsline{toc}{chapter}{Introdução}

Este relatório parcial retrata o estado atual do trabalho de conclusão de curso em questão. Segundo o \nameref{chap:crono} do trabalho, aprovado em proposta de projeto, ele compreende o que foi feito dentre os meses de agosto e outubro de 2016.

O Capítulo \nameref{chap:parciais} e suas respectivas seções contém descrições dos passos já concluídos e resultados obtidos atee então. Dada a natureza essencialmente computacional do trabalho, eles decorrem diretamente das implementações e do código fonte apresentado no Apêndice \ref{apen:fonte}, que são citados ao longo do texto à medida que se faz necessária a referência.

No Capítulo \nameref{chap:posterior} é apresentada uma previsão tentada dos estágios imediatamente posteriores, sob uma ótica de planejamento. 

Os conjuntos de dados utilizados no trabalho encontram-se no Anexo \ref{anex:dados}.

O problema de identificação de \textit{clusters}\footnote{Termo traduzido usualmente como \textbf{conglomerados}} aqui estudado consiste em, a partir de um \textit{Espaço Geográfico} \textbf{G}, que por sua vez é dividido em \textbf{regiões}, e onde ocorre um evento aleatório, dicotômico e não-estocástico, encontrar combinações destas regiões (uma \textbf{zona}), onde o evento ocorra com probabilidade significativamente maior.


% ----------------------------------------------------------
% Capitulo de textual  
% ----------------------------------------------------------
\chapter{Resultados Parciais}
\label{chap:parciais}
\section{\textit{Scan} Circular de Kulldorff}

Este algoritmo encontra-se implementado no \textit{software} SaTScan \cite{_satscan_????}, mantido pelo próprio Martin Kulldorff. Apesar de gratuito (i.e. distribuído sem ônus para o usuário), sua licença não é \textit{livre} na maioria das acepções utilizadas \cite{_free_????}, mais gravemente a não disponibilização de seu código-fonte. Isso impede, por exemplo, que utilizemos como base a implementação do SaTScan, feita em Java, para o R, o \textit{software} escolhido para implementação neste trabalho. Uma pesquisa no serviço \textit{crantastic} \cite{_its_????}, que disponibiliza uma ferramenta de busca nos pacotes disponíveis livremente para utilização no R, não retornou resultados para palavras-chave como "kulldorff circular scan", "circular scan" e similares. No \textit{github}, uma espécie de rede-social de programadores onde são disponibilizados sob licenças livres projetos das mais diversas áreas, também não retornou dentre resultados implementações utilizáveis.

Não obstante estas limitações, uma implementação inédita do algoritmo serviu para facilitar grandemente a compreensão do mesmo, e de sua limitação que levou à proposição da nova heurística que serve como Objetivo deste trabalho. Ainda que o método sugerido por Kulldorff seja facilmente explicável em algumas palavras, colocá-lo em prática a partir do zero possibilitou uma familiarização importante.

O código-fonte em \ref{anex:scan-circular} é, apesar de correto, problemático em termos de eficiência do código. Uma análise de perfil (\textit{profiling}) do código utilizando a ferramenta \texttt{Rprofvis} do R, disponível em \cite{_rpubs_????}, mostra o culpado: o incremento iterativo do \textit{data frame} \texttt{resultado}. Este tipo de erro é comum, sendo inclusive citado no text "The R Inferno" \cite{burns2012r}, do consultor Patrick Burns. Na verdade, sua solução é, em teoria, simples: via de regra a alocação de uma estrutura de dados qualquer previamente considerando seu tamanho final ou máximo sempre será mais eficiente que sua alocação iterativa. Como o objetivo maior do trabalho não envolve uma implementação eficiente do \textit{Scan} Circular, não há inicialmente a preocupação em mitigar o efeito da técnica flagrantemente ineficaz de alocação de dados. 

Ainda sobre o código-fonte, sua leitura deve ser razoavelmente intuitiva mas ele essencialmente especifica uma função que realiza duas coisas, a saber: 

\begin{itemize}
\item Obtém a matriz de distâncias dos centroides das regiões.
\item Calcula o logaritmo da razão de verossimilhança para cada uma das zonas candidatas.
\end{itemize}

Estas zonas são nada mais do que composições de regiões, e como o nome sugere e deve ficar claro à leitura do código-fonte a heurística de composição destas zonas é a distância a partir de um centroide inicial até os centroides das outras regiões no espaço.

\newpage
\section{Simulação de Monte-Carlo}

Na etapa seguinte, foi feita a verificação de significância do \textit{cluster} mais verossímil, i.e. cujo logaritmo da razão de verossimilhança seja maior. A hipótese nula a ser testada é de que a probabilidade de ocorrência do evento \textit{dentro} da zona candidata que apresenta o maior valor da estatística em questão é igual à probabilidade de ocorrência do evento fora desta zona em particular.

Para tanto, podemos gerar um número \textit{considerável} de realizações de uma distribuição Multinomial sob hipótese nula, ou seja, cujas componentes do seu vetor de probabilidade são proporcionais ao produto entre a razão do total de casos sobre a população total e a população na região. Em notação usual, foram geradas realização de uma variável aleatória $X \sim (\sum_i P_i, \mu_0 P_1, \mu_0 P_2, \dots, \mu_n P_n)$, onde $\mu_0$ é a média geral sob a hipótese nula, e $P_i, i=1,\dots, n$ é a população na região i.

De acordo com o conjunto de dados aqui utilizado, disponibilizado pelo orientador, chegou-se à conclusão de que a zona formada pelas regiões 130, 131, 143, 144, 145, 157, 158 têm probabilidade de ocorrência do evento significativamente superior à observada nas outras zonas. 

O baixo número de realizações da simulação dá-se em parte por conta da falta de eficiência do \textit{Scan} Circular retratada na seção anterior. Ainda assim, o resultado observado no trabalho condiz com o resultado obtido previamente pelo orientador.



\chapter{Próximos Passos}
\label{chap:posterior}

Há a tentativa de implementação de uma heurística para composição das zonas que não leve em consideração a distância entre os centroides, mas sim quais apresentam a propriedade de adjacência entre si. Este algoritmo estaria descrito, grosso modo, nos seguintes termos.

\begin{enumerate}
\item Iniciliza-se uma região i qualquer.
\item A partir de i, e tendo acesso a quais regiões são adjacentes a i, obter a zona onde o logaritmo da razão de verossimilhança é maior quando há a composição da região i com suas adjacentes.
\item A partir da zona composta pelas duas regiões, i e aquela que mais aumenta a razão de verossimilhança, compor um vetor de vizinhanças para a zona a partir das regiões que a compõem.
\item Pare quando o tamanho populacional da zona for maior que metade da população total.
\end{enumerate}

Eventualmente, o procedimento descrito acima encontrará um certo número de \textit{zonas candidatas}. A partir destas zonas candidatas, a intenção é trabalhar com zonas candidatas que apresentem interseção entre si e compô-las de forma que possam apresentar formatos não-regulares.

\chapter{Cronograma}
\label{chap:crono}

As atividades a serem desenvolvidas são as seguintes:

\begin{enumerate}
	\item \label{etapa1} Escolha do tema a ser abordado.
	\item \label{etapa2} Desenvolvimento da proposta de projeto.
	\item \label{etapa3} Entrega da proposta de projeto.
	\item \label{etapa4} Revisão de literatura.
	\item \label{etapa5} Elaboração da apresentação da proposta.
	\item \label{etapa6} Apresentação oral da proposta.
	\item \label{etapa7} Implementação
    \item \label{etapa8} Verificação dos Modelos
	\item \label{etapa9} Elaboração do relatório parcial.
	\item \label{etapa10} Entrega do relatório parcial ao Professor Orientador.
	\item \label{etapa11} Correção do relatório parcial.
	\item \label{etapa12} Entrega do relatório parcial para a banca.
	\item \label{etapa13} Desenvolvimento do modelo.
    \item \label{etapa14} Elaboração do relatório final.
	\item \label{etapa15} Entrega do relatório final ao Professor Orientador.
	\item \label{etapa16} Correção do do relatório final.
	\item \label{etapa17} Entrega do relatório final para a banca.
\end{enumerate}

\definecolor{midgray}{gray}{.5}
\def\tablename{Tabela }%
\begin{table}[!htbp]\scriptsize
\centering {\caption{Cronograma}
		\begin{tabular}{|c|c|c|c|c|c|c|c|c|c|c|c|c|}
		\hline
		&\multicolumn{5}{c|}{2016}&\multicolumn{7}{c|}{2017}\\
		\hline
		&Agosto&Setembro&Outubro&Novembro&Dezembro&Janeiro&Fevereiro&Março&Abril&Maio&Junho&Julho\\
		\hline
		\ref{etapa1}&\cellcolor{midgray}&&&&&&&&&&&\\
		\hline
		\ref{etapa2}&&\cellcolor{midgray}&&&&&&&&&&\\
		\hline	
		\ref{etapa3}&&\cellcolor{midgray}&&&&&&&&&&\\
		\hline			
		\ref{etapa4}&&\cellcolor{midgray}&\cellcolor{midgray}&&&&&&&&&\\
		\hline	
		\ref{etapa5}&&&\cellcolor{midgray}&&&&&&&&&\\
		\hline
		\ref{etapa6}&&&\cellcolor{midgray}&\cellcolor{midgray}&&&&&&&&\\
		\hline	
		\ref{etapa7}&\cellcolor{midgray}&\cellcolor{midgray}&\cellcolor{midgray}&\cellcolor{midgray}&\cellcolor{midgray}&&&&&&&\\
		\hline	
		\ref{etapa8}&&&\cellcolor{midgray}&&&\cellcolor{midgray}&&&&&&\\
		\hline	
		\ref{etapa9}&&&&&\cellcolor{midgray}&&&&&&&\\
		\hline	
		\ref{etapa10}&&&&&&\cellcolor{midgray}&&&&&&\\
		\hline	
		\ref{etapa11}&&&&&&\cellcolor{midgray}&\cellcolor{midgray}&\cellcolor{midgray}&&&&\\
		\hline	
		\ref{etapa12}&&&&&&&&\cellcolor{midgray}&\cellcolor{midgray}&&&\\
		\hline	
		\ref{etapa13}&&&&&&&&\cellcolor{midgray}&\cellcolor{midgray}&\cellcolor{midgray}&&\\
        \hline
        \ref{etapa14}&&&&&&&&&\cellcolor{midgray}&\cellcolor{midgray}&\cellcolor{midgray}&\\
        \hline
        \ref{etapa15}&&&&&&&&&&\cellcolor{midgray}&&\\
        \hline
        \ref{etapa16}&&&&&&&&&&\cellcolor{midgray}&\cellcolor{midgray}&\\
        \hline	
        \ref{etapa17}&&&&&&&&&&&&\cellcolor{midgray}\\
        \hline
		\end{tabular}}
\end{table}


% ----------------------------------------------------------
% ELEMENTOS PÓS-TEXTUAIS
% ----------------------------------------------------------
\postextual

% ----------------------------------------------------------
% Referências bibliográficas
% ----------------------------------------------------------
\bibliography{proj}

% ----------------------------------------------------------
% Glossário
% ----------------------------------------------------------
%
% Consulte o manual da classe abntex2 para orientações sobre o glossário.
%
%\glossary

% ----------------------------------------------------------
% Apêndices
% ----------------------------------------------------------

% ---
% Inicia os anexos
% ---
\begin{apendicesenv}

% Imprime uma página indicando o início dos anexos
\partanexos


% ---
\chapter{Códigos-fonte}
\label{apen:fonte}

Os códigos-fonte aqui disponibilizados encontram-se  hospedados no repositório \url{https://github.com/august-o/tcc}.


\section{\textit{Scan} Circular de Kulldorff}
\label{anex:scan-circular}
% ---
\lstinputlisting{llr_ksc.R}


% ---
\section{Simulação de Monte-Carlo}
\lstinputlisting{sim.ksc.R}

\end{apendicesenv}

\begin{anexosenv}



\chapter{Conjuntos de Dados}
\label{anex:dados}

Os conjuntos de dados utilizados neste trabalho encontram-se hospedados no repositório \url{https://github.com/august-o/tcc/tree/master/dados_hex}. 

\section{dados.csv}

Arquivo tabulado que descreve o número de casos observados e a população de 203 regiões, bem como as coordenadas de seus \textit{centroides}.

\section{hex.adj}

Matriz de adjacências para as 203 regiões. A componente (i,j) nesta matriz é igual a 1 se as regiões i e j são adjacentes entre si e 0 caso contrário.
\end{anexosenv}

\phantompart

\printindex


\end{document}
